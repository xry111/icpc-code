\section{数论}

\subsection{中国剩余定理}

解方程组 $x \equiv ans[i] \ (mod \ d[i])$。

\lstinputlisting[language=c++]{number-theory/crt.cc}

\subsection{线性筛}

\lstinputlisting[language=c++]{number-theory/linear-sieve.cc}

\subsection{BSGS}

已知 $a, b, M$,$gcd(a, M) = 1$,解 $a^x \equiv b \ (mod \ M)$。

\lstinputlisting[language=c++]{number-theory/bsgs.cc}

\subsection{扩展 BSGS}

不要求 $gcd(a, M) = 1$。

\lstinputlisting[language=c++]{number-theory/extend-bsgs.cc}

\subsection{Wilson 定理}

$(n-1)! \equiv -1 \ (mod \ n)$,当且仅当 $n$ 是质数。
例如,可以计算组合数取模。

\lstinputlisting[language=c++]{number-theory/wilson.cc}

\subsection{整除分块}

对于这样的东西

$$\sum_{i=l}^r f(i) g(\lfloor n/i \rfloor)$$

这样的东西,容易看出 $\lfloor n/i \rfloor$ 具有的不同数值的个数是很少的。
给定一个 $i$,我们可以算出最小的,使得 $\lfloor n/i' \rfloor >
\lfloor n/i \rfloor$ 的 $i'$:

$$i' = \lfloor n / \lfloor \frac{n}{d} \rfloor \rfloor + 1$$

这就可以分出 $\mathcal{O}(\sqrt{n})$ 块,每一块里面 $g$ 的值都相同,
只要能对 $f$ 快速求区间和,就能用 $\mathcal{O}(\sqrt{n})$ 的时间算出来。

代码可以参考下面的杜教筛中的求和代码。

\subsection{杜教筛}

用于求 $\sum_{i=1}^n f(i)$,要求 $f(i)$ 是积性的。

原理是首先需要将 $f(x)$ 卷一个 $g(x)$,得到

$$s = \sum_{i=1}^n \sum_{d|i} f(x/d) g(d) =
\sum_{d=1}^n \sum_{k=1}^{\lfloor n/d \rfloor} f(k) g(d) =
\sum_{k=1}^n f(k) + \sum_{d=2}^n g(d) \sum_{k=1}^{\lfloor n/d \rfloor}f(k)$$

可见,要求的东西就是

$$s - \sum_{d=2}^n g(d) \sum_{k=1}^{\lfloor n/d \rfloor} f(k)$$

如果 $s$ 和 $g$ 可以比较快地求出来,
就可以后面的那个东西进行整除分块,递归地算 $f(x)$ 的前缀和。
这样做时间复杂度是 $\mathcal{O}(n^{3/4})$ 的,
如果用线性筛预处理前 $\mathcal{O}(\sqrt{n})$ 个前缀和,
则可以做到 $\mathcal{O}(n^{2/3})$。

下面的例子是筛 $\phi$ 的前缀和。

\lstinputlisting[language=c++]{number-theory/apiadu-sieve.cc}
